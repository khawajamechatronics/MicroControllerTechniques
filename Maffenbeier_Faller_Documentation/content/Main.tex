\section{Concept}

Since this \textit{Tetris} game consists of more than one screen
(\texttt{Menu}, \texttt{Game}, \texttt{Highscore})
the game is split up into multiple source files.
This minimizes internal dependencies and enables a modularized project.
The concept of modules is also applied to all driver files for each peripheral
(See Figure~\ref{fig:overview}).

Each driver listens for events an calls a corresponding handler function if an
event occurred.
Since the handler might change through different screens a callback function
is specified which is internally represented as function-pointer.
With this the dynamic dispatch of events is possible since the function-pointer
can be swapped arbitrary at any time.

\begin{figure}[!h]
  \begin{center}
    \includegraphics[width=\linewidth]%
      {images/Overview.pdf}
  \end{center}
  \caption{Global overview of the project}
  \label{fig:overview}
\end{figure}

\pagebreak

\section{Main Loop}

The main concept of the game loop is to enter \textit{Low-Power Mode 0}
(\texttt{LP0}) whenever the game field is steady.
When the tetromino drops or the user inputs a command (key-press) the
\textit{CPU} is woken up and processes the game field.
After sending all characters to the \textit{UART} interface the sleep mode
is entered again (See Figure~\ref{fig:main_loop}).

\begin{figure}[!h]
  \begin{center}
    \includegraphics[width=\linewidth]%
      {images/MainLoop.pdf}
  \end{center}
  \caption{Main loop of the game}
  \label{fig:main_loop}
\end{figure}

\section{The Game}

To view the source code of the game open this \textit{PDF} file with
\textit{WinRAR} or a compatible extractor (It is a \textit{RAR} file).